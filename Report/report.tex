\documentclass[a4paper,11pt]{article}
\usepackage[T1]{fontenc}
\usepackage[utf8]{inputenc}
\usepackage{lmodern}
\usepackage{hyperref}
\usepackage{graphicx}
\usepackage{rotating}
\usepackage{listings}
\usepackage{color}
\usepackage{appendix}
\usepackage{caption}
\usepackage{float}
\usepackage{listings}

% Swedish
\usepackage[swedish]{babel}

% Table of contents depth 3 levels: A.B.C
\setcounter{tocdepth}{3}

\lstset{ %
language=C,                  	% the language of the code
basicstyle=\tiny,       	% the size of the fonts that are used for the code
numbers=left,                   % where to put the line-numbers
numberstyle=\tiny,      	% the size of the fonts that are used for the line-numbers
stepnumber=5,                   % the step between two line-numbers. If it's 1, each line will be numbered
numbersep=5pt,                  % how far the line-numbers are from the code
backgroundcolor=\color{white},  % choose the background color. You must add \usepackage{color}
showspaces=false,               % show spaces adding particular underscores
showstringspaces=false,         % underline spaces within strings
showtabs=false,                 % show tabs within strings adding particular underscores
frame=single,                   % adds a frame around the code
tabsize=2,                      % sets default tabsize to 2 spaces
captionpos=t,                   % sets the caption-position to top
breaklines=true,                % sets automatic line breaking
breakatwhitespace=false,        % sets if automatic breaks should only happen at whitespace
title=\lstname,                 % show the filename of files included with \lstinputlisting; also try caption instead of title
}


\begin{document}

\title{{\huge Tweet till Elektronisk Dörrskylt} \\
	Kandidatarbete D/IT 2012 \\}
\author{Andreas Åkesson, Anton Svensson, Fredrik Brosser, Jakob Kallin, Kim Burgestrand, Lars Tidstam \\ \\
   	Chalmers Tekniska Högskola \\ \\
	\begin{tabular}{l c r}
		\texttt{frebro} & \texttt{@} & \texttt{student.chalmers.se}\\
		\texttt{frebro} & \texttt{@} & \texttt{student.chalmers.se}\\
		\texttt{frebro} & \texttt{@} & \texttt{student.chalmers.se}\\
		\texttt{frebro} & \texttt{@} & \texttt{student.chalmers.se}\\
		\texttt{frebro} & \texttt{@} & \texttt{student.chalmers.se}\\
		\texttt{frebro} & \texttt{@} & \texttt{student.chalmers.se}\\\\
	\end{tabular}
	}
	
\maketitle
\thispagestyle{empty}
\pagebreak

\thispagestyle{empty}
	\tableofcontents
\thispagestyle{empty}

\pagebreak

\setcounter{page}{1}
\section{Inledning}

\end{document}
